\chapter{Fundamenta\c{c}\~{a}o Te\'{o}rica}
\label{cap:fundamentacao teorica}


\section{Sistema Embarcado}

Segundo \textcite{Wolf:2001} um sistema de computação embarcado é qualquer dispositivo que inclui um computador programável, o qual não é direcionado para resolver problemas de propósito geral, mas sim uma situação ou necessidade específica, operando geralmente com recursos limitados e executando algoritmos para resposta em tempo real.

A diferenciação de um sistema embarcado de um sistema de propósito geral se dá por algumas características:
\begin{itemize}
	\item \emph{Tempo real}: a resposta do sistema deve ser dada em um tempo limite. Caso a resposta venha fora do limite de tempo aceitável, o sistema pode ser tornar instável, parar de funcionar ou realizar ações fora do previsto gerando danos catastróficos. 
	\item \emph{Diferentes taxas de entrada de dados}: um sistema embarcado pode receber diversas entradas de dados com velocidades diferentes, ele deve tratar esta situação e oferecer uma saída sincronizada e sem perdas.
	\item \emph{Custo de produção}: todos os componentes de hardware devem ser escolhidos para minimizar o custo de produção, por isso, um projeto cuidadosamente elaborado se faz necessário.
	\item \emph{Consumo de energia}: conjunto hardware/software dever ser projetado de maneira tal que o consumo de recursos energéticos seja sempre o mínimo possível. Um elevado consumo influi tanto no custo de produção  quanto no custo de operação.
\end{itemize}

Um exemplo de sistema embarcado a se considerar é um roteador wireless conforme a figura 2.1, pois possui as seguintes características:

\begin{itemize}
	\item É um computador de propósito específico: realiza funcões de roteamento, controle de conexões, controle do sinal de rádio e implementa seviços da camada física a camada de rede (modelo OSI);
	\item Opera em tempo real: todos os pacotes que passam pelo roteador deve ser processados em tempo hábil, caso contrário a comunicação entre dispositivos que estão conectados a ele se torna impraticável;
	\item Custo de produção resuzido: São utilizados processadores de custo reduzido, quantidade de memória limitada e outros componentes de menor custo a fim de reduzir o preço total e facilitar a produção em massa.
	\item Baixo consumo de energia: O conjunto hardware/software opera com baixíssimo consumo de energia a fim de não impactar na conta de energia do consumidor, afinal é um produto que funciona, em geral, 24 horas por dia.
\end{itemize}

\begin{figure}[h!]
			\centering
			\includegraphics[width=0.4\textwidth]{figures/embarcado_exemplo.jpg}
			\caption{Um roteador wireless é um sistema embarcado}
			\label{1}
	\end{figure}


\hfill

\section{Projeto de Sistemas Embarcados}

Modelos para projeto de sistemas embarcados têm sido propostos ao longo dos anos, como é caso do Embedded UML. De acordo com \textcite{Martin:2001}, autor do modelo, este é um profile UML que representa a síntese de várias ideias discutidas pela comunidade que adota UML no projeto de aplicações de tempo real. 

Foram selecionadas as melhores práticas de análise e especificação de requisitos para sistemas os quais necessitam que hardware e software sejam co-projetados, introduzido um conceito de mapeamento da plataforma para gerar uma implementação otimizada de hardware e software, uma vez que conforme \textcite{Wolf:2001}, o custo de produção e o consumo de recursos energéticos deve ser o mínimo possível.

Esta metodologia fornece uma abordagem que permite o desenvolvimento de ferramentas de análise automatizada, simulação, síntese e geração de código equanto se define um padrão UML para embarcados.

O próprio \textcite{Martin:2001}, propõe em um segundo artigo, após utilizar UML para modelagem de sistemas embarcados, novas extensões UML que cobrem lacunas não atendidas pela linguagem, como suporte a modelagem de hardware baseada em VHDL (que é uma abordagem dependente de plataforma).

É claro que na UML existem algumas limitações de análise de cenários com requisitos não funcionais, que no caso de embarcados têm prioridade igual a dos requisitos funcionais e de suma importância para a confiabilidade do sitema. Esta limitação é descrita por \textcite{Espinoza:2008} que propõem práticas complementares para analisar estes requisitos não funcionais, a fim de tornar os sistemas mais eficientes de confiáveis.

\textcite{Wolf:2001} propõe uma abstração em alto nível dos principais passos no projeto de sistemas embarcados, são estes do topo à base:

\begin{itemize}
	\item \emph{Requisitos}: onde são obtidas informações sobre o que se desenvolver no sistema, obter os requisitos esperados pelo cliente/usuário e extrair apenas o que for importante para o projeto levando requisitos funcionais e não funcionais são levados em conta.
	\item \emph{Especificação}: etapa onde se realiza a mapeamento dos requisitos funcionais de um sistema embarcado, quais funções ele deve fornecer ao usuário. Os Casos de Uso da UML são um meio de documentar os requisitos.
	\item \emph{Projeto Arquitetural}: nesta etapa é feita a quebra do sistema em componentes de alto nível e como eles se integram e se comunicam. A UML oferece o Diagrama de Componentes para esta finalidade.
	\item \emph{Projeto de componentes de hardware e software}: para cada componente definido, são mapeados seus atributos e métodos, levando a uma abordagem utilizando Orientação a Objetos. O Diagrama de Classes da UML permite fazer este tipo de representação.
	 \item \emph{Integração do sistema}: após a construção de cada componente de hardware e implementação dos componentes de software, estes são integrados seguindo o projeto.
\end{itemize}
\textcite{Marwedel:2001}  inclui etapas de teste dentro das etapas de projeto, designando o fluxo como modelo-V. As etapas de teste são: 

\begin{itemize}
	\item \emph{Testes unitários}: Teste individual para cada unidade mínima funcional do sistema, com isto é possível verificar o comportamento de cada parte e, em caso de defeitos, corrigi-los antes de qualquer integração, poupando tempo de desenvolvimento uma vez que seria dispendioso, depois do sistema pronto, rastrear e econtrar o problema.
	\item \emph{Testes de integração}: Verificação da comunicação entre os diferentes componentes quanto interligados. Esta etapa é válida pois mesmo que os componentes funcionem individualmente e separados, sempre que se integram partes um novo comportamento é gerado, o que pode levar a erros inesperados.
	\item \emph{Aceitação e Uso}: Verifica se o comportamento do sistema está dentro do esperado nas especificações, é um teste do sistema completo e em funcionamento, parar confirmação que todos os requisitos elicitados estão sendo atendidos.
\end{itemize}



\subsection{Professor Orientador}

O professor orientador tem a função de ajudar o aluno no direcionamento do seu trabalho, sem, entretanto, desenvolver partes desse 
trabalho para o aluno. O orientador, apenas sugere caminhos que o aluno deverá seguir, acompanha seu trabalho, motivando e corrigindo eventuais erros.

Antes de apresentar o TCC-I ou TCC-II, o aluno deve submetê-lo previamente, \textbf{e obrigatoriamente}, à apreciação de seu orientador. 
Dado o aval do mesmo, a proposta poderá ser encaminhada e apresentada para avaliação.

Compete ao Professor Orientador:

	\begin{itemize}
		\item Informar ao professor da disciplina a linha de pesquisa que irá atuar
	
		\item Orientar a elaboração do Trabalho de Conclusão
	
		\item Auxiliar o aluno na resolução de problemas conceituais, técnicos e de relacionamento decorrentes da atividade
	
		\item Estabelecer o plano e cronograma de trabalho em conjunto com o orientando
	
		\item Informar o orientando sobre as normas, procedimentos e critérios de avaliação respectivos
	
		\item Informar ao aluno, caso haja atraso no cronograma de trabalho ou o não cumprimento das orientações, se o trabalho tem condições ou não de ser 
		encaminhado para avaliação
	
		\item liberar o trabalho para que haja a apresentação do aluno bem como informar ao professor da disciplina quanto à apresentação do aluno (Anexo C)
	
		\item Rubricar as 3 (três) vias encaminhadas para avaliação (TCC-I ou TCC-II) quando estiver ciente e de acordo, 
		conforme suas orientações, do material entregue
	
		\item Presidir a banca examinadora do trabalho por ele orientado
	
		\item Comunicar ao professor da disciplina situações que exijam providências, assim que ocorrerem.
	\end{itemize}


\subsection{Co-orientador}

	\begin{itemize}	
		\item Será solicitado formalmente pelo orientador a Professor da disciplina e será designado por esse para atender questão específica do trabalho
		
		\item Trabalhará em conjunto ao orientador e desempenhará papel solicitado pelo mesmo.
	\end{itemize}


\subsection{Orientando}

Compete ao Orientando:

	\begin{itemize}
		\item Comparecer às reuniões marcadas pelo professor da disciplina sobre o Trabalho de Conclusão
		
		\item Escolher a temática a ser trabalhada em consonância com as linhas de pesquisa do curso
		
		\item Contatar professor para definir orientador e informar a Coordenação do Projeto (entrega do Anexo-B)
		
		\item Cumprir as datas limites determinadas no calendário de atividades do TCC. O não cumprimento dos prazos será penalizado com perda de pontuação; 
		(0,2 pts, na média final, por dia de atraso na entrega da monografia)
		
		\item Comparecer às orientações sobre o trabalho; o não comparecimento de três (03) orientações seguidas implica em reprovação por falta
		
		\item Seguir as orientações do professor designado à orientação
		
		\item Cumprir o plano e o cronograma de trabalho elaborado em conjunto com o professor-orientador
		
		\item Comunicar ao professor da disciplina toda e qualquer situação que possa comprometer, de alguma forma, o processo de elaboração, bem como, a conclusão do 			trabalho o quanto antes, para que a coordenação possa analisar o ocorrido e tomar as providências cabíveis
		
		\item Comparecer perante a banca na data, hora e local estabelecido para a realização da sessão de avaliação do TCC
	\end{itemize}


\subsection{Os Acompanhamentos de Orientação}

As reuniões de orientação deverão ser documentadas conforme modelo presente no Anexo A e serão entregues ao professor da disciplina no dia da entrega da carta (ANEXO A) solicitando defesa de TCC.
Tanto professor orientador como orientando deverão ter uma cópia dos acompanhamentos de orientação.
	

\subsection{A Banca Examinadora}

A banca examinadora do TCC-I e TCC –II, deverá ser composta por, no mínimo, 3 professores. A banca será constituída pelo 
professor orientador e por dois outros professores. Se houver co-orientação, o professor co-orientador pode compor a banca, contudo sua avaliação não computará nota para o alunosua avaliação não computará nota para o aluno 

Os membros da banca examinadora poderão sugerir alterações no trabalho (parte escrita e/ou implementação). Para o TCC-I as alterações deverão ser feitas, com o acompanhamento do orientador, para que sejam incluídas no trabalho e avaliadas no TCC-II. Para o TCC-II, estas deverão ser feitas até duas semanas depois da apresentação (ver data limite), supervisionadas pelo professor-orientador, para constar no(s) volume(s) final(is) do TCC, que ficará à disposição na biblioteca

O volume final para arquivamento (TCC-II) só será aceito pela coordenação de TCC se estiver validado pelo professor orientador, indicando sua concordância com o conteúdo do mesmo, e a assinatura do aluno


\subsection{Seminários}
Conforme calendário os seminários destinados aos alunos matriculados em TCC, abordam temas que auxiliarão na elaboração do documento escrito e na defesa.

Serão e seminários:
	\begin{itemize}
		\item Seminário I – Estrutura do Trabalho de Conclusão de Curso

		\item Seminário II – Normas ABNT

		\item Seminário III – Apresentação do TCC (defesa e material)
	\end{itemize}

A participação do aluno nos seminários é um dos critérios que consta na ata de avaliação. O não comparecimento acarretará perda de 0,2 pt por seminário.


\subsection{As Datas Limite}

As datas limites serão estabelecidas e divulgadas de acordo com o calendário acadêmico de cada período acadêmico.


\subsection{Nota Final}

Para aprovação do aluno no TCC, o mesmo deverá:

	\begin{itemize}
		\item Atender à exigência da frequência mínima de 75$'%'$ (setenta e cinco) às orientações. 
		A frequência do aluno será validada a partir do formulário de acompanhamento de reunião de orientação os quais devem ser preenchidos a cada acompanhamento, 			pelo orientador e pelo aluno

		\item Obter, no mínimo, grau 6,0 (seis). Este grau será composto pela média aritmética das avaliações dos membros da banca examinadora. Cada membro da banca 			receberá uma planilha com itens a avaliar (por notas). Ao término da defesa será preenchida uma ata final de avaliação de TCC constando a média final do aluno.
		Caso o aluno não alcance grau mínimo seis (6,0) deverá matricular-se novamente na disciplina para desenvolver novamente o trabalho ou concluir o 			desenvolvimento do mesmo
	\end{itemize}


\section{Trabalho de Conclusão de Curso I (TCC-I)}

O aluno, em parceria com um professor orientador, deve delimitar um tema, a ser abordado, dentro das linhas de pesquisa do curso.

Deve então dar início à documentação de seu trabalho elaborando uma monografia com os capítulos contendo a fundamentação teórica e modelagem da implementação do trabalho proposto. Ao final do semestre defendê-lo à uma banca examinadora.

\subsection{Estrutura do TCC-I}

Na monografia, o aluno deverá documentar seu trabalho para ser arquivado e, no futuro, referenciado por outras pessoas, lembrando sempre que Trabalho de Conclusão de Curso deve ser escrito tendo em vista uma metodologia científica. 

A monografia deve seguir a seguinte estrutura: 
	
	\indent \textbf{Capa} \\
	\indent \textbf{Folha de Rosto} \\
	\indent \textbf{Ficha para Catalogação} (deve ser impressa no verso da folha de rosto) \\
	\indent \textbf{Epígrafe} (opcional) \\
	\indent \textbf{Dedicatória} (opcional) \\
	\indent \textbf{Agradecimentos} (opcional) \\
	\indent \textbf{Resumo} \\
	\indent \textbf{Abstract} (resumo em inglês) \\
	\indent \textbf{Sumário} \\
	\indent \textbf{Lista de Figuras} (opcional) \\
	\indent \textbf{Lista de Tabelas} (opcional) \\
	\indent \textbf{Lista de Abreviaturas e siglas} \\
	\indent \textbf{Introdução} \\
	\indent \textbf{Desenvolvimento} \\
	\indent \textbf{Conclusão} \\
	\indent \textbf{Referências Bibliográficas} \\
	\indent \textbf{Obras Consultadas} \\
	\indent \textbf{Anexos e/ou Apêndices} (opcional) \\\\

\textbf{Introdução} - é o primeiro capítulo da monografia. Apresenta o contexto do trabalho proposto com a definição do problema, os objetivos (geral e específicos), os motivos que levaram à decisão de se abordar o tema e a organização do trabalho.

\textbf{Desenvolvimento} - corresponde aos demais capítulos da monografia, que descrevem sobre o tema proposto, revisão da literatura, metodologia aplicada, ferramentas e modelagem (se aplicável) do trabalho a ser implementado.

\textbf{Conclusão} - como o TCC-I é o início da monografia não será possível uma conclusão, portanto devem ser apresentadas as dificuldades encontradas, até o momento no trabalho, e resultados esperados do trabalho proposto.

A formatação (margens, espaçamentos, citações, paginação, etc.) de todo o documento, deve estar voltada para um trabalho científico, portanto, 
os alunos devem seguir o modelo de monografia adotado pelo curso e disponível no site do mesmo. 
Solicitamos ainda aos alunos que utilizem as obras abaixo:

	\begin{itemize}	
		\item  FURASTÉ, Pedro Augusto. Normas Técnicas para o Trabalho Científico (Nova ABNT). 14ª edição. Porto Alegre, 2006.

		\item SILVA, Edna Lúcia da. Metodologia da Pesquisa e Elaboração de Dissertação – 3ª ed. rev. e atual. – Florianópolis: Laboratório de Ensino a Distância da 			UFSC, 2001.

		\item BRASIL, ABNT – Associação Brasileira de Normas Técnicas. NBR 14724.

		\item BRASIL, ABNT – Associação Brasileira de Normas Técnicas. NBR 10520.

		\item BRASIL, ABNT – Associação Brasileira de Normas Técnicas. NBR 6023.
	\end{itemize}
	

\subsection{Avalição do TCC-I}

O TCC-I deverá ser apresentado perante uma banca examinadora a ser definida pelo professor da disciplina, para a qual o aluno apresentará seu trabalho, desde a justificativa do problema que o levou a desenvolvê-lo até as discussões do material levantado.

O aluno terá 25 (vinte e cinco) minutos para defesa de sua proposta, onde utilizará os recursos audiovisuais que achar necessário e serão utilizados mais 10 (dez) minutos para responder aos questionamentos de cada membro da banca avaliadora. A banca será constituída pelo professor orientador (presidente) e por dois outros professores, podendo ser um convidado externo. Avaliado o trabalho escrito e “ouvidas” as sugestões da banca, o aluno deverá fazer as modificações necessárias.

No TCC-I caso o aluno não alcance grau mínimo 6,0 (seis) deverá matricular-se novamente na disciplina pra desenvolver novamente o trabalho (ou concluir o desenvolvimento do mesmo)  em TCC-I.
É necessário que o aluno seja aprovado em TCC-I para a conclusão do trabalho em TCC-II. 


\section{Trabalho de Conclusão de Curso II (TCC-II)}

O aluno deverá por em prática a modelagem apresentada em TCC-I além de concluir a monografia (implementação, resultados obtidos, etc.). Haverá nova defesa da documentação e demonstração do que foi desenvolvido (implementação) de acordo com esta documentação.


\subsection{Estrutura do TCC-II}

Apresenta a mesma estrutura do TCC-I. Entretanto, no TCC-II, o aluno irá complementar a monografia de acordo com as solicitações feitas pela banca examinadora (na defesa do TCC-I) e com tópicos relacionados à sua implementação. 

A monografia deve seguir a seguinte estrutura: 

	\indent \textbf{Capa} \\
	\indent \textbf{Folha de Rosto} \\
	\indent \textbf{Ficha para Catalogação} (deve ser impressa no verso da folha de rosto) \\
	\indent \textbf{Epígrafe} (opcional) \\
	\indent \textbf{Dedicatória} (opcional) \\
	\indent \textbf{Agradecimentos} (opcional) \\
	\indent \textbf{Resumo} \\
	\indent \textbf{Abstract} (resumo em inglês) \\
	\indent \textbf{Sumário} \\
	\indent \textbf{Lista de Figuras} (opcional) \\
	\indent \textbf{Lista de Tabelas} (opcional) \\
	\indent \textbf{Lista de Abreviaturas e siglas} \\
	\indent \textbf{Introdução} \\
	\indent \textbf{Desenvolvimento} \\
	\indent \textbf{Conclusão} \\
	\indent \textbf{Referências Bibliográficas} \\
	\indent \textbf{Obras Consultadas} \\
	\indent \textbf{Anexos e/ou Apêndices} (opcional) \\\\


\textbf{Introdução} - é o primeiro capítulo da monografia. Apresenta o contexto do trabalho proposto com a definição do problema, os objetivos (geral e específicos), os motivos que levaram à decisão de se abordar o tema e a organização do trabalho.

\textbf{Desenvolvimento} - corresponde aos demais capítulos da monografia, que descrevem sobre o tema proposto, revisão da literatura, metodologia aplicada, ferramentas e modelagem (se aplicável) do trabalho a ser implementado.

\textbf{Conclusão} - se os objetivos foram atingidos, dificuldades encontradas e sugestões para trabalhos futuros.

A formatação (margens, espaçamentos, citações, paginação, etc.) de todo o documento, deve estar voltada para um trabalho científico, portanto, 
os alunos devem seguir o modelo de monografia adotado pelo curso e disponível no site do mesmo. 
Solicitamos ainda aos alunos que utilizem as obras abaixo:

	\begin{itemize}	
		\item  FURASTÉ, Pedro Augusto. Normas Técnicas para o Trabalho Científico (Nova ABNT). 14ª edição. Porto Alegre, 2006.

		\item SILVA, Edna Lúcia da. Metodologia da Pesquisa e Elaboração de Dissertação – 3ª ed. rev. e atual. – Florianópolis: Laboratório de Ensino a Distância da 			UFSC, 2001.

		\item BRASIL, ABNT – Associação Brasileira de Normas Técnicas. NBR 14724.

		\item BRASIL, ABNT – Associação Brasileira de Normas Técnicas. NBR 10520.

		\item BRASIL, ABNT – Associação Brasileira de Normas Técnicas. NBR 6023.
	\end{itemize}


\subsection{Avaliação do TCC-II}

O TCC-II deverá ser apresentado perante uma banca examinadora a ser definida pelo professor da disciplina, para a qual o aluno apresentará seu trabalho, desde a justificativa do problema que o levou a desenvolvê-lo até as discussões do material levantado e a conclusão.

Esta apresentação deverá ser, necessariamente, oral e descritiva, onde o aluno deverá também na parte oral resumir as principais funções do sistema, o modo como será usado na organização ou ambiente e os seus benefícios.

Para esta apresentação oral, o aluno deverá preparar o que irá falar e utilizar recursos didáticos, considerando o tempo de 45 minutos; cada membro da banca avaliadora terá 10 minutos para questionamentos.

Não serão aceitas justificativas para a não demonstração das implementações, implicando assim em reprovação.

As apresentações dos TCCs são abertas ao público interessado. Sugere-se, fortemente, que os alunos de TCC-I assistam às bancas de seus colegas de TCC-II, como experiência.

Caso o aluno não alcance a nota mínima de 6,0 pontos em TCC-II, deverá matricular-se, novamente na disciplina, no próximo semestre.
	
A nota do aluno só será lançada mediante entrega de 2 cópias da versão revisada com visto do orientador, já incluindo as modificações sugeridas pela banca, no formato final (impresso em jato de tinta ou laser), em capa dura, na cor preta, bem como um CD com a versão final do trabalho e os produtos resultantes da pesquisa, quando for o caso.

	
