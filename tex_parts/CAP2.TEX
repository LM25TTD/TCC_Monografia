\chapter{Fundamenta\c{c}\~{a}o Te\'{o}rica}
\label{cap:fundamentacao teorica}


\section{Sistema Embarcado}

Segundo \textcite{Wolf:2001} um sistema de computação embarcado é qualquer dispositivo que inclui um computador programável, o qual não é direcionado para resolver problemas de propósito geral, mas sim uma situação ou necessidade específica, operando geralmente com recursos limitados e executando algoritmos para resposta em tempo real.

A diferenciação entre um sistema embarcado de um sistema de propósito geral se dá por algumas características:
\begin{itemize}
	\item \emph{Tempo real}: a resposta do sistema deve ser dada em um tempo limite. Caso a resposta venha fora do limite de tempo aceitável, o sistema pode ser tornar instável, parar de funcionar ou realizar ações fora do previsto gerando danos catastróficos. 
	\item \emph{Diferentes taxas de entrada de dados}: um sistema embarcado pode receber diversas entradas de dados com velocidades diferentes, ele deve tratar esta situação e oferecer uma saída sincronizada e sem perdas.
	\item \emph{Custo de produção}: todos os componentes de hardware devem ser escolhidos para minimizar o custo de produção, por isso, um projeto cuidadosamente elaborado se faz necessário.
	\item \emph{Consumo de energia}: conjunto hardware/software dever ser projetado de maneira tal que o consumo de recursos energéticos seja sempre o mínimo possível. Um elevado consumo influi tanto no custo de produção  quanto no custo de operação.
\end{itemize}

Um exemplo de sistema embarcado a se considerar é um roteador wireless conforme a figura 2.1, pois possui as seguintes características:

\begin{itemize}
	\item É um computador de propósito específico: realiza funcões de roteamento, controle de conexões, controle do sinal de rádio e implementa seviços da camada física a camada de rede (modelo OSI);
	\item Opera em tempo real: todos os pacotes que passam pelo roteador deve ser processados em tempo hábil, caso contrário a comunicação entre dispositivos que estão conectados a ele se torna impraticável;
	\item Custo de produção resuzido: São utilizados processadores de custo reduzido, quantidade de memória limitada e outros componentes de menor custo a fim de reduzir o preço total e facilitar a produção em massa.
	\item Baixo consumo de energia: O conjunto hardware/software opera com baixíssimo consumo de energia a fim de não impactar na conta de energia do consumidor, afinal é um produto que funciona, em geral, 24 horas por dia.
\end{itemize}

\begin{figure}[h!]
			\centering
			\includegraphics[width=0.4\textwidth]{figures/embarcado_exemplo.jpg}
			\caption{Um roteador wireless é um sistema embarcado}
			\label{1}
	\end{figure}


\hfill

\section{Projeto de Sistemas Embarcados}

Modelos para projeto de sistemas embarcados têm sido propostos ao longo dos anos, como é caso do Embedded UML. De acordo com \textcite{Martin:2001}, autor do modelo, este é um profile UML que representa a síntese de várias ideias discutidas pela comunidade que adota UML no projeto de aplicações de tempo real. 

Foram selecionadas as melhores práticas de análise e especificação de requisitos para sistemas os quais necessitam que hardware e software sejam co-projetados, introduzido um conceito de mapeamento da plataforma para gerar uma implementação otimizada de hardware e software, uma vez que conforme \textcite{Wolf:2001}, o custo de produção e o consumo de recursos energéticos deve ser o mínimo possível.

Esta metodologia fornece uma abordagem que permite o desenvolvimento de ferramentas de análise automatizada, simulação, síntese e geração de código equanto se define um padrão UML para embarcados.

O próprio \textcite{Martin:2001}, propõe em um segundo artigo, após utilizar UML para modelagem de sistemas embarcados, novas extensões UML que cobrem lacunas não atendidas pela linguagem, como suporte a modelagem de hardware baseada em VHDL (que é uma abordagem dependente de plataforma).

É claro que na UML existem algumas limitações de análise de cenários com requisitos não funcionais, que no caso de embarcados têm prioridade igual a dos requisitos funcionais e de suma importância para a confiabilidade do sitema. Esta limitação é descrita por \textcite{Espinoza:2008} que propõem práticas complementares para analisar estes requisitos não funcionais, a fim de tornar os sistemas mais eficientes de confiáveis.

\textcite{Wolf:2001} propõe uma abstração em alto nível dos principais passos no projeto de sistemas embarcados, são estes do topo à base:

\begin{itemize}
	\item \emph{Requisitos}: onde são obtidas informações sobre o que se desenvolver no sistema, obter os requisitos esperados pelo cliente/usuário e extrair apenas o que for importante para o projeto levando requisitos funcionais e não funcionais são levados em conta.
	\item \emph{Especificação}: etapa onde se realiza a mapeamento dos requisitos funcionais de um sistema embarcado, quais funções ele deve fornecer ao usuário. Os Casos de Uso da UML são um meio de documentar os requisitos.
	\item \emph{Projeto Arquitetural}: nesta etapa é feita a quebra do sistema em componentes de alto nível e como eles se integram e se comunicam. A UML oferece o Diagrama de Componentes para esta finalidade.
	\item \emph{Projeto de componentes de hardware e software}: para cada componente definido, são mapeados seus atributos e métodos, levando a uma abordagem utilizando Orientação a Objetos. O Diagrama de Classes da UML permite fazer este tipo de representação.
	 \item \emph{Integração do sistema}: após a construção de cada componente de hardware e implementação dos componentes de software, estes são integrados seguindo o projeto.
\end{itemize}
\textcite{Marwedel:2001}  inclui etapas de teste dentro das etapas de projeto, designando o fluxo como modelo-V. As etapas de teste são: 

\begin{itemize}
	\item \emph{Testes unitários}: Teste individual para cada unidade mínima funcional do sistema, com isto é possível verificar o comportamento de cada parte e, em caso de defeitos, corrigi-los antes de qualquer integração, poupando tempo de desenvolvimento uma vez que seria dispendioso, depois do sistema pronto, rastrear e econtrar o problema.
	\item \emph{Testes de integração}: Verificação da comunicação entre os diferentes componentes quanto interligados. Esta etapa é válida pois mesmo que os componentes funcionem individualmente e separados, sempre que se integram partes um novo comportamento é gerado, o que pode levar a erros inesperados.
	\item \emph{Aceitação e Uso}: Verifica se o comportamento do sistema está dentro do esperado nas especificações, é um teste do sistema completo e em funcionamento, parar confirmação que todos os requisitos elicitados estão sendo atendidos.
\end{itemize}



\section{Microcontroladores}

Um microcontrolador é uma unidade de processamento que, diferente de um microprocessador, traz todos os módulos necessários ao seu funcionamento (como memória volátil, memória somente leitura, blocos de entrada e saída, conversores analógico-digital e digital-analógico, linhas para troca de dados com componentes externos) no interior de um único chip, podendo assim ser programado para executar uma rotina de propósito específico.

Realizando uma comparativa com os microprocessadores, pode-se constatar que os microcontroladores operam em uma frequência muito baixa no entanto adequada a maioria das aplicações as quais são designados, como por exemplo: controlar uma esteira, uma caldeira ou uma máquina. Devido a esta característica, o consumo de energia é pequeno, normalmente na casa dos miliwatts, além disso os microncontroladores podem entrar em modo de espera, aguardando por um evento externo como o pressionar de uma tecla ou acionamento de um sensor.

Uma outra característica antagônica em relação aos microprocessadores é que, enquanto neste últimos é feito um superdimensionamento de recursos sendo limitado pela faixa financeira que o usuário pode investir, em um projeto com microcontroladores o superdimensionamento é um erro de projeto, um desperdício de recurso que reflete diretamente no preço do equipamento final, sendo multiplicado no caso de uma produção em larga escala.

As aplicações deste componente compreendem principalmente automação e controle de produto periféricos, como sistemas de controle de motores automotivos, máquinas industriais, de escritório e residenciais, brinquedos, sistemas de supervisão e outros. Por reduzir o tamanho, custo e consumo de energia se comparados ao microprocessadores, se tornam uma alternativa eficiente para controlar processos e aplicações. A figura 2.2 mostra um microcontrolador da fabricante Atmel, pode-se observar que ele possui um único encapsulamento com todos os periféricos necessários ao seu funcionamento embutidos.


\begin{figure}[h!]
			\centering
			\includegraphics[width=0.4\textwidth]{figures/microcont_atm.jpg}
			\caption{Microcontrolador Atmel}
			\label{1}
	\end{figure}


\section{Plataforma Arduino}

Arduino é uma plataforma para prototipagem eletrônica que utiliza o conceito de hardware livre. Neste conceito deve-se prover um lançamento irrestrito de informações sobre o projeto de hardware como digramas eletrônicos, estrutura de produtos, layout de placas de circuito impresso, rotinas de baixo nível e qualquer outra informação necessária para um construção from-scratch do projeto.

A linhagem mais utilizada da placa microcontroladora Arduino (modelos Duemilanove e Uno) foi concebida com base em um microncontrolador Atmel AVR montado em placa única, possui suporte de entra/saída embutido, uma linguagem de programação padrão baseada em C/C++.

A plataforma surgiu para que amadores, entusiastas e outras pessoas que não teriam acesso ao controladores e ferramentas mais sofisticadas, pudessem criar seus projetos, tornando-os acessíveis, flexíveis e com baixo custo. Uma placa Arduino, em geral, possui os seguintes elementos em sua construção:

\begin{itemize}
	\item Um microcontrolador
	\item Placa base com reguladores de voltagem adequados ao microcontrolador.
	\item Linhas de Entrada/Saída digitais e analógicas.
	\item Linhas de comunicação serial.
	\item Interface USB para programação e interação com um computador hospedeiro.
\end{itemize} 

Seguindo estas características comuns, ao decorrer da existência do projeto foram desenvolvidas diversas placas microncontroladoras, cada uma carrega um codinome e especificações de hardware próprias. A tabela 2.1 descreve algumas placas da família Arduino.

\begin{table}[!h]
	\begin{tabular}{|r|c|c|c|c|c|}
		\hline 
			\textbf{Modelo} & \textbf{Clock} & \textbf{E/S Digital} & \textbf{E/S Analógico} & \textbf{Alimentação} & \textbf{Flash}
		\\
		\hline
			Arduino Due & 84MHz & 54 & 12 & 7-12V & 512 KB
		\\
		\hline
			Arduino Leonardo & 16MHz & 20 & 12 & 7-12V & 32 KB
		\\
		\hline
			\textbf{Arduino Uno} & 16MHz & 14 & 6 & 7-12V & 32 KB
		\\
		\hline
			Arduino Duemilanove & 16MHz & 14 & 6 & 7-12V & 32 KB
		\\
		\hline
			Arduino Pro & 8MHz & 14 & 6 & 3.3-12V & 32 KB
		\\
		\hline
			Arduino Mega & 16MHz & 54 & 16 & 7-12V & 256 KB
		\\
		\hline
			Arduino Mini 05 & 16MHz & 14 & 6 & 7-9V & 32 KB
		\\
		\hline
			Arduino Fio & 8MHz & 14 & 8 & 3.3-12V & 32 KB
		\\
		\hline
			LilyPad Arduino & 8MHz & 14 & 6 & 2.7-5.5V & 32 KB
		\\


\hline
\end{tabular}
\caption{Modelos de placa Arduino}
\end{table}

Uma possibilidade bastante interessante para projetos que utilizam a plataforma Arduino é a expansão da placa microcontroladora através da agregação de novos hardwares, estes chamados de shields (do inglês concha).

\subsection{Arduino Shields}

Placas Arduinos e outras baseadas no projeto utilizam expansões de hardware chamadas shields. São placas de circuito impresso fixadas ao topo da placa microcontroladora através dos pinos de conexão e se comunicam com a unidade principal através dos canais de Entrada/Saída analógicos ou digitais, ou ainda, através do canal de comunicação serial.

Caso não utilizem a mesma pinagem, pode-se empilhar diversos shields na mesma placa controladora. A idéia destas expansões é a de criar componentes com determinada especialização, muito similar ao conceito de framework em software. Existem shields para as mais diversas aplicações como:

\begin{itemize}
	\item Conectividade em rede Ethernet
	\item Conectividade em rede Zigbee
	\item Conectividade em rede Wireless 802.11
	\item Controle de motores
	\item Conexão com rede celular GPRS
	\item Funcionalidade GPS
	\item Middleware Android
	\item Controle de Relés
	\item Bluetooh
	\item Sintetizador de Voz
\end{itemize}
