\appendix
\chapter{Anexos}
\label{cap:anexos}

\section{Descrição dos casos de uso}

\begin{flushleft}
\tablefirsthead{}
\tablehead{}
\tabletail{}
\tablelasttail{}
\tablecaption{Caso de Uso Manter Usuário}
\begin{supertabular}{|m{5.5680003cm}|m{9.325cm}|}
\hline
Nome do caso de uso &
Manter Usuário\\\hline
Sumário &
Caso que descreve as etapas para manutenção de um usuário no sistema através da figura do Administrador, envolvendo as operação CRUD e associações com outros elementos do sistema.\\\hline
Ator primário &
Administrador\\\hline
Atores secundários &
Usuário comum\\\hline
Precondições &
Usuário com CPF válido.\\\hline
Fluxo Principal &
1. Administrador deve se autenticar no sistema web.

2. Acessar página de gerenciamento de Usuários.

3. Escolher opção de cadastrar novo usuário.

4. Preencher os dados com informações válidas.

5. Confirmar ação e finalizar processo.\\\hline
Fluxo Alternativo &
(1) Credenciais inválidas

a. O sistema retorna uma mensagem de erro informando o ocorrido e impossibilitando acesso sem autenticação.

(3) Buscar Usuário

a. Selecionar os filtros de busca e os valores para cada um deste.

b. O usuário ou lista de usuários compatíveis com os filtros são retornados.

(3) Apagar Usuário

a. Selecionar usuários que se deseja remover através de um componente de seleção múltipla.

b. O sistema retorna o nome dos usuários a serem apagados, solicitando confirmação.

c. O administrador confirma a ação.

(3) Atualizar dados do Usuário

a. Selecionar usuário a sofrer o processo.

b. Página de cadastro de usuário é preenchida com os dados atualmente persistidos.

c. Preencher os dados com informações válidas.

d. Confirmar ação e finalizar processo.

(5) Falha na validação dos dados

a. O sistema retornará mensagem informando quais campos possuem valor inválido.
\\\hline
Pós-condições &
Usuário estará cadastrado no sistema, podendo assim acessá-lo e executar as ações descritas em seus casos de uso. Em caso de remoção, os dados relacionados ao usuário removido também serão apagados.\\\hline
Requisitos não funcionais &
Design minimalista e funcional da página de inserção de usuário, validação de valores dos campos, double-check para remoção de usuários.\\\hline
Autor &
Leandro Bentes\\\hline
Data &
17/05/2013\\\hline
\end{supertabular}
\end{flushleft}

\begin{flushleft}
\tablefirsthead{}
\tablehead{}
\tabletail{}
\tablelasttail{}
\topcaption{Caso de Uso Manter ModVeicular}
\begin{supertabular}{|m{5.5680003cm}|m{9.325cm}|}
\hline
Nome do caso de uso &
Manter ModVeicular\\\hline
Sumário &
Caso que descreve os processos de manutenção de um módulo veicular no sistema. O módulo veicular é um componente acoplado ao veículo que permite a este ser rastreado e bloqueado via web.\\\hline
Ator primário &
Administrador\\\hline
Atores secundários &
~
\\\hline
Precondições &
Existir módulo físico com a numeração serial única e válida.\\\hline
Fluxo Principal &
1. Administrador deve se autenticar no sistema web.

2. Acessar página de gerenciamento de Módulos Veiculares.

3. Escolher opção de cadastrar novo módulo.

4. Preencher os dados com informações válidas.

5. Confirmar ação e finalizar processo.\\\hline
Fluxo Alternativo &
(1) Credenciais inválidas

a. O sistema retorna uma mensagem de erro informando o ocorrido e impossibilitando acesso sem autenticação.

(2) Buscar Módulo Veicular

a. Selecionar os filtros e os valores para cada um deste.

b. O módulo ou lista de módulos veiculares compatíveis com os filtros são retornados.

(2) Atualizar dados do Módulo Veicular

a. Buscar módulo veicular.

b. Selecionar módulo a sofrer o processo, a página de cadastro deve surgir preenchida com os valores atuais de cadastro deste.

c. Preencher os dados com informações válidas.

d. Confirmar ação e finalizar processo.

(2) Remover módulo veicular

a. Selecionar equipamentos que se deseja remover através de um componente de seleção múltipla.

b. O sistema retorna o número serial dos equipamentos a serem apagados, solicitando confirmação.

c. O administrador confirma a ação.

d. O sistema remove o(s) equipamento(s) desejados.

(5) Falha na validação dos dados

a. O sistema retornará mensagem informando quais campos possuem valor inválido, para que o administrador possa corrigir facilmente em caso de engano.

b. Caso o equipamento a ser cadastrado já exista no banco de dados, o sistema informará esta situação com mensagem apropriada. \\\hline
Pós-condições &
Equipamento estará cadastrado no sistema pronto para sofrer interação.\\\hline
Requisitos não funcionais &
Design minimalista e funcional da página de Inserção de Equipamento, validação de valores dos campos.\\\hline
Autor &
Leandro Bentes\\\hline
Data &
14/04/2013\\\hline
\end{supertabular}
\end{flushleft}

\begin{flushleft}
\tablefirsthead{}
\tablehead{}
\tabletail{}
\tablelasttail{}
\topcaption{Caso de Uso Associar Modulo-Usu\'{a}rio}
\begin{supertabular}{|m{5.5680003cm}|m{9.325cm}|}
\hline
Nome do caso de uso &
Associar Modulo-Usuário\\\hline
Sumário &
Descrição da associação entre o equipamento e seu dono (um usuário cadastrado), possibilitando operações-chave como rastreamento e bloqueio. \\\hline
Ator primário &
Administrador\\\hline
Atores secundários &
Usuário comum\\\hline
Precondições &
Usuário e equipamentos envolvidos previamente cadastrados.\\\hline
Fluxo Principal &
1. Administrador deve se autenticar no sistema web.

2. Acessar página de Associação de Equipamentos, a lista de equipamentos livres é exibida.

3. O administrador seleciona um equipamento, uma tela de pesquisa de usuários é mostrada.

4. O usuário para associação é mostrado.

5. O administrador deve confirmar a associação que é efetivada no sistema.\\\hline
Fluxo Alternativo &
(1) Credenciais inválidas

a. O sistema retorna uma mensagem de erro informando o ocorrido e impossibilitando acesso sem autenticação.

(3) Caso executado a partir do cadastro de Usuário.

a. A tela de pesquisa de usuários não é mostrada, uma vez que a associação será feita com o usuário que está sendo cadastrado. \\\hline
Pós-condições &
O usuário que se associou ao equipamento pode rastreá-lo, e bloquear veículo no qual ele está instalado.

Equipamento não poderá ser associado a outro usuário sem que se remova a atual associação.\\\hline
Requisitos não funcionais &
Não definidos ainda\\\hline
Autor &
Leandro Bentes\\\hline
Data &
10/04/2013\\\hline
\end{supertabular}
\end{flushleft}

\begin{flushleft}
\tablefirsthead{}
\tablehead{}
\tabletail{}
\tablelasttail{}
\topcaption{Caso de Uso Atualizar Pr\'{o}prio Cadastro}
\begin{supertabular}{|m{5.5680003cm}|m{9.325cm}|}
\hline
Nome do caso de uso &
Atualizar Próprio Cadastro\\\hline
Sumário &
Caso que descreve a atualização dos próprios dados, feita por um usuário acessando o sistema.\\\hline
Ator primário &
Usuário comum, Administrador\\\hline
Atores secundários &
~
\\\hline
Precondições &
~
\\\hline
Fluxo Principal &
1. Usuário deve se autenticar no sistema web.

2. Acessar página Meus Dados.

3. O formulário é preenchido com os dados atualmente persistidos.

4. O usuário altera os campos desejados e aciona opção salvar.

5. O sistema valida os valores e os dados são persistidos.\\\hline
Fluxo Alternativo &
(1) Credenciais inválidas

a. O sistema retorna uma mensagem de erro informando o ocorrido e impossibilitando acesso sem autenticação.

(5) Falha na validação dos dados

a. O sistema retornará mensagem informando quais campos possuem valor inválido, para que o usuário possa corrigir facilmente em caso de engano.\\\hline
Pós-condições &
Modificação nos dados iniciais do Usuário.\\\hline
Requisitos não funcionais &
Design minimalista e funcional da página de Inserção de Usuário, validação de valores dos campos.\\\hline
Autor &
Leandro Bentes\\\hline
Data &
14/04/2013\\\hline
\end{supertabular}
\end{flushleft}

\begin{flushleft}
\tablefirsthead{}
\tablehead{}
\tabletail{}
\tablelasttail{}
\topcaption{Caso de Uso Manter Ve\'{i}vulos}
\begin{supertabular}{|m{5.5680003cm}|m{9.325cm}|}
\hline
Nome do caso de uso &
Manter Veículos\\\hline
Sumário &
Caso que descreve as etapas para manutenção de veículos do usuário no sistema, envolvendo as operações CRUD.\\\hline
Ator primário &
Usuário comum\\\hline
Atores secundários &
~
\\\hline
Precondições &
Veículo com placa e código de chassi válido.\\\hline
Fluxo Principal &
1. Usuário deve se autenticar no sistema web.

2. Acessar página de gerenciamento de Veículos.

3. Escolher opção de cadastrar novo veículo.

4. Preencher os dados com informações válidas.

5. Confirmar ação e finalizar processo.\\\hline
Fluxo Alternativo &
(1) Credenciais inválidas

a. O sistema retorna uma mensagem de erro informando o ocorrido e impossibilitando acesso sem autenticação.

(3) Buscar Veículo

a. Selecionar os filtros de busca e os valores para cada um deste.

b. O veículo ou lista de veículos compatíveis com os filtros são retornados.

(3) Apagar Veículo

a. Selecionar os veículos que se deseja remover através de um componente de seleção múltipla.

b. O sistema retorna marca, modelo e placa dos veículos a serem apagados, solicitando confirmação.

c. O usuário confirma a ação.

(3) Atualizar dados do Veículo

a. Após uma busca, selecionar o veículo a sofrer o processo.

b. Página de cadastro de veículo é preenchida com os dados atualmente persistidos.

c. Preencher os dados com informações válidas.

d. Confirmar ação e finalizar processo.

(5) Falha na validação dos dados

a. O sistema retornará mensagem informando quais campos possuem valor inválido, para que o usuário possa corrigir facilmente em caso de engano.

b. Caso o veículo a ser cadastrado já exista no banco de dados, o sistema informará esta situação com mensagem apropriada. \\\hline
Pós-condições &
Veículo estará cadastrado no sistema, podendo ser associado a algum equipamento para quer o rastreio e/ou bloqueio sejam permitidos. Em caso de remoção, o módulo associado ao veículo será marcado como livre, somente podendo sofrer bloqueio e rastreamento caso seja associado a um novo.\\\hline
Requisitos não funcionais &
Design minimalista e funcional da página de inserção de veículo, validação de valores dos campos, double-check para remoção de veículo.\\\hline
Autor &
Leandro Bentes\\\hline
Data &
19/05/2013\\\hline
\end{supertabular}
\end{flushleft}

\begin{flushleft}
\tablefirsthead{}
\tablehead{}
\tabletail{}
\tablelasttail{}
\topcaption{Caso de Uso Listar Módulos}
\begin{supertabular}{|m{5.5680003cm}|m{9.325cm}|}
\hline
Nome do caso de uso &
Listar Modulos\\\hline
Sumário &
Caso que descreve os processos de manutenção dos módulos veiculares pertencentes ao usuário. \\\hline
Ator primário &
Usuário\\\hline
Atores secundários &
~
\\\hline
Precondições &
Usuário cadastrado e equipamentos associados a ele.\\\hline
Fluxo Principal &
1. Usuário deve se autenticar no sistema web.

2. Acessar página Meus Módulos ou nome correspondente.

3. Todos os módulos pertencentes ao usuário são retornados, podendo associá-lo ou desassociá-lo de um veículo.

4. Usuário filtra o resultado para encontrar o(s) módulo(s) desejado(s).\\\hline
Fluxo Alternativo &
(1) Credenciais inválidas

a. O sistema retorna uma mensagem de erro informando o ocorrido e impossibilitando acesso sem autenticação.

(3) Detalhar módulo

a. Usuário filtra o resultado para encontrar o(s) módulo(s) desejado(s).

b. Selecionar opção Detalhar Módulo.

(3) Usuário sem módulo pertencente a ele.

a. O sistema retornará mensagem informando que o usuário não possui módulos cadastrados.

(4) Filtro sem correspondência

a. Caso o usuário aplique um filtro no resultado que não encontre qualquer módulo correspondente, o sistema informará que não existem resultados para o filtro aplicado.\\\hline
Pós-condições &
Possibilidade de associar módulo a um veículo livre.\\\hline
Requisitos não funcionais &
~\\\hline
Autor &
Leandro Bentes\\\hline
Data &
15/04/2013\\\hline
\end{supertabular}
\end{flushleft}

\section{Códigos, esquemas e diagramas}

Um dois principais algoritmos gerados neste projeto foi o firmware do módulo veicular embarcado. Utilizando uma combinação de bibliotecas de interface com os shields e funções implementadas de acordo com os diagramas de atividades das figuras \ref{fig:seqsetup} e \ref{fig:seqloop} foi gerada a rotina embarcada.

Os códigos \ref{code:embedded1}, \ref{code:embedded2}, \ref{code:embedded3} e \ref{code:embedded4} apresentam o firmware em linguagem C/C++ gerado neste projeto.

\renewcommand{\baselinestretch}{0.5}  % distância entre linhas
\begin{codigo}[!htb]
\fontsize{9pt}{9pt}\selectfont
      \begin{boxit}  % coloca o código dentro de um Box
      \vspace{2mm}
      \VerbatimInput[xleftmargin=8mm,numbers=left,obeytabs=true]{sources/embedded.cpp}
   \end{boxit}
   \caption{\it Código do Módulo Embarcado Parte 1}
   \label{code:embedded1}
\end{codigo}

\renewcommand{\baselinestretch}{0.5}  % distância entre linhas
\begin{codigo}[!htb]
\fontsize{9pt}{9pt}\selectfont
      \begin{boxit}  % coloca o código dentro de um Box
      \vspace{2mm}
      \VerbatimInput[xleftmargin=8mm,numbers=left,obeytabs=true]{sources/embedded2.cpp}
   \end{boxit}
   \caption{\it Código do Módulo Embarcado Parte 2}
   \label{code:embedded2}
\end{codigo}

\renewcommand{\baselinestretch}{0.5}  % distância entre linhas
\begin{codigo}[!htb]
\fontsize{9pt}{9pt}\selectfont
      \begin{boxit}  % coloca o código dentro de um Box
      \vspace{2mm}
      \VerbatimInput[xleftmargin=8mm,numbers=left,obeytabs=true]{sources/embedded3.cpp}
   \end{boxit}
   \caption{\it Código do Módulo Embarcado Parte 3}
   \label{code:embedded3}
\end{codigo}

\renewcommand{\baselinestretch}{0.5}  % distância entre linhas
\begin{codigo}[!htb]
\fontsize{9pt}{9pt}\selectfont
      \begin{boxit}  % coloca o código dentro de um Box
      \vspace{2mm}
      \VerbatimInput[xleftmargin=8mm,numbers=left,obeytabs=true]{sources/embedded4.cpp}
   \end{boxit}
   \caption{\it Código do Módulo Embarcado Parte 4}
   \label{code:embedded4}
\end{codigo}


\afterpage{% Insert after the current page
\clearpage
\KOMAoptions{paper=A3,paper=landscape,pagesize}
\recalctypearea

\begin{figure}[p]
   \includegraphics[width=1.0\textwidth]{figures/esquema_arduino.png}%
   \caption{Esquema el\'{e}trico da placa Arduino Uno}
\end{figure}

\clearpage

\begin{figure}[!htb]
	\centering
	\includegraphics[width=1.1\textwidth]{figures/6_web_manager.png}
	\caption{Modelo de classes do WebManager}
	\label{fig:classgeralwebman}
\end{figure}

\clearpage

\begin{figure}[!htb]
			\centering
			\includegraphics[width=0.99\textwidth]{figures/schemaGPS.png}
			\caption{Esquema elétrico do GPS Shield}
			\label{fig:esquemagpdshield}
\end{figure}
\clearpage

\recalctypearea
}