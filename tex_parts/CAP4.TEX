\chapter{Resultados e Discussões}
\label{cap:resultados e discussoes}

\section{Módulo Web (WebManager) - Testes de Carga}

A fim de validar o comportamento do aplicação Web durante um alto número de requisições, caso que ocorre quando muitos módulos embarcados enviam requests HTTP no mesmo espaço de tempo (simulando uma situação de crescimento da plataforma), a ferramenta para testes de performance Apache JMeter foi utilizada para gerar o tráfego e o software VisualVM serviu para coletar métricas da aplicação como consumo de memória e uso da CPU.

Uma quantidade de 200 módulos concorrendo aos recursos do serviro foi aplicada, gerando resultados satisfatórios, com o consumo de memória em picos de 750 MB com gráfico de comportamento serrilhado, ou seja, liberação de memória não utilizada e ausência de leaks, além disso o consumo máximo de CPU ficou em torno de 30 \%, uma margem segura. A figura 16 mosta o comportamento serrilhado do gráfico de consumo de memória, demonstrando a ausência de leaks e atuação correta do Garbage Collector.

\begin{figure}[!htb]
	\centering
	\includegraphics[width=15.00cm\textwidth]{figures/teste_heap.png}
	\caption{Gráfico do consumo de memória heap}
	\label{Figura 16}
\end{figure}

O consumo de CPU mesmo com a ação do Garbage Collector ficou em margens coerentes de acordo com o gráfico da figura 17.

\begin{figure}[!htb]
	\centering
	\includegraphics[width=15.00cm\textwidth]{figures/teste_cpu.png}
	\caption{Gráfico do uso de CPU}
	\label{Figura 17}
\end{figure}

\section{Resultados}

Os testes do protótipo incluem aplicação do módulo veicular em um automóvel para validar seu comportamento. Foi utilizado um veículo da marca Fiat modelo Palio Fire ano 2002, no qual o módulo embarcado foi acoplado no circuito de alimentação da bomba de combustível para efetuar o corte de alimentação no caso da solicitação de bloqueio, resultando no desligamento do veículo. Os testes de rastreamento com automotor em movimento, bloqueio e desbloqueio foram realizados com sucesso e registrados em vídeo. A figura 18 mostra o protótipo integrado ao circuito elétrico do automóvel de testes.

\begin{figure}[!htb]
	\centering
	\includegraphics[width=12.00cm\textwidth]{figures/integracao.jpg}
	\caption{Integração do protótipo}
	\label{Figura 18}
\end{figure}

Procedeu-se a coleta de dados que foram armazenados em banco relacional. As seções de coleta mostram um espaço aproximado de um minuto entre as inserções na base de dados, tempo previsto pois os parâmetros de delay setados no firmware do módulo embarcado foram calculados para resultar neste intervalo. A figura 19 mostra as linhas inseridas no banco de dados durante o período de testes.

\begin{figure}[!htb]
	\centering
	\includegraphics[width=12.00cm\textwidth]{figures/dados_coletados.png}
	\caption{Dados coletados}
	\label{Figura 19}
\end{figure}

\vspace{200pt}


\section{Próximos Passos}

\begin{itemize}
	\item Finalizar a implementação da UI.
	\item Verificar a possibilidade de implementação de caminho feito pelo veículo.
	\item Verificar possibilidade de comunicação assíncrona entre módulo e webmanager.
	\item Melhorar eficiência do uso do módulo GSM.	
\end{itemize}




\section{Algoritmos}

\renewcommand{\baselinestretch}{0.5}  % distância entre linhas
\begin{codigo}[htb]
\fontsize{9pt}{9pt}\selectfont
      \begin{boxit}  % coloca o código dentro de um Box
      \vspace{2mm}
      \VerbatimInput[xleftmargin=8mm,numbers=left,obeytabs=true]{sources/embedded.cpp}
   \end{boxit}
   \caption{\it Código do Módulo Embarcado Parte 1}
   \label{code:embedded}
\end{codigo}

\renewcommand{\baselinestretch}{0.5}  % distância entre linhas
\begin{codigo}[htb]
\fontsize{9pt}{9pt}\selectfont
      \begin{boxit}  % coloca o código dentro de um Box
      \vspace{2mm}
      \VerbatimInput[xleftmargin=8mm,numbers=left,obeytabs=true]{sources/embedded2.cpp}
   \end{boxit}
   \caption{\it Código do Módulo Embarcado Parte 2}
   \label{code:embedded2}
\end{codigo}

\renewcommand{\baselinestretch}{0.5}  % distância entre linhas
\begin{codigo}[htb]
\fontsize{9pt}{9pt}\selectfont
      \begin{boxit}  % coloca o código dentro de um Box
      \vspace{2mm}
      \VerbatimInput[xleftmargin=8mm,numbers=left,obeytabs=true]{sources/embedded3.cpp}
   \end{boxit}
   \caption{\it Código do Módulo Embarcado Parte 3}
   \label{code:embedded3}
\end{codigo}

\renewcommand{\baselinestretch}{0.5}  % distância entre linhas
\begin{codigo}[htb]
\fontsize{9pt}{9pt}\selectfont
      \begin{boxit}  % coloca o código dentro de um Box
      \vspace{2mm}
      \VerbatimInput[xleftmargin=8mm,numbers=left,obeytabs=true]{sources/embedded4.cpp}
   \end{boxit}
   \caption{\it Código do Módulo Embarcado Parte 4}
   \label{code:embedded4}
\end{codigo}