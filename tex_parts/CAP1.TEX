\chapter {Introdu\c{c}\~{a}o}
\label{cap:introducao}

\section{Descri\c{c}\~{a}o do Problema}

De acordo dados do \textcite{denatran:2011} a quantidade de roubos e furtos de veículos no Brasil vem avançando de forma 
significativa durante os últimos anos, aumentando proporcionalmente ao número 
de automóveis em circulação. O texto sintetiza informações das secretarias estaduais de 
segurança pública e dos Detrans, mostrando as cidades do Rio de Janeiro, São Paulo e Salvador ocupando, respectivamente, os três primeiros lugares em número de roubos e furtos no país.

Ainda segundo o \textcite{denatran:2011}, em média, 150 carros são 
roubados ou furtados por dia no Rio de Janeiro, quatro por dia em São Paulo e 
2,8 por dia em Salvador. Para a polícia, os grupos que comandam o esquema de
roubo e furto articulam as ações de dentro de presídios e possuem ramificações
em vários estados. 

A principal utilidade dos automóveis sinistrados é a transformação em dublês (ou clones), passando por alterações no chassi, placas e outros dados para venda ilegal em outros estados. Uma parte também serve para a prática de crimes, como o transporte de quadrilhas durante assaltos e o transporte de drogas.

O maior problema relacionado ao crime é o risco de vida que os ocupantes do veículo são expostos, principalmente quando se há uma tentativa de reação contra os criminosos, que ocorre pela incerteza se o bem perdido será recuperado pelas autoridades e em boas condições.

Diante deste cenário, é necessária uma solução que permita a geolocalização em tempo real contando com mecanismos de bloqueio que podem ser controlados remotamente, o que geraria tranquilidade aos que sofrerem roubo/furto dos seus veículos, eliminando a ânsia de reação em um primeiro momento.

\hfill

\section{Trabalhos Relacionados}	

Durante o levantamento bibliográfico realizado para obtenção de trabalhos correlatos foram encontrados três projetos de relevância na categoria de alarme automotivo inteligente, alguns utilizando inclusive comunicação remota inter-dispositivos e hardware open-source.

\textcite{ivan:2007} descreve um projeto de sistema de alarme automotivo 
cuja principal funcionalidade é a detecção de um evento sonoro no interior do veículo e a sinalização a um microcontrolador. O microcontrolador, por sua vez, está programado para acionar um dispositivo que realiza uma chamada telefônica destinada ao aparelho celular do proprietário do veículo de modo a alertá-lo sobre a anomalia. O objetivo geral é a detecção do choro de uma criança ou barulho de um animal esquecido dentro do veículo. 

Em toda extensão do trabalho de \textcite{ivan:2007} verificou-se uma abordagem bastante prática, com 
experimentos, construção de circuitos, exibição do funcionamento do 
protótipo e um fundamental teórico bastante sucinto 
que se focou principalmente nos tipos de microfone. O ponto negativo ficou 
por conta da ausência de modelagem do problema, não existem diagramas UML ou 
similares que representem as funcionalidades, projeto de classes e sequência, 
que são essenciais para qualidade do projeto.

Em sua monografia, \textcite{leandro:2010} propõe um sistema de segurança veicular para ser integrado a um alarme já existente no veículo, este sistema objetiva, além de avisar ao proprietário via SMS/GPRS quando o alarme for acionado, detectar quando algum objeto, criança ou animal de estimação for esquecido sobre o banco verificando a presença do peso sobre este. 

O projeto se baseia na plataforma Arduino, utilizando a filisofia de hardware 
livre, modularização de componentes e desenvolvimento simplificado de protótipos 
para posterior fabricação, o autor optou pela utilização de uma placa Arduino 
Mega com módulo externo. 

Foram disponibilizados os passos realizados no projeto, um modelo esquemático de testes além de anexar o código fonte aplicado na placa controladora, todavia há a ausência de esquemas de modelagem, principalmente baseados em UML, além da falta do esquema eletrônico geral, o que facilitaria bastante a análise das limitações e aplicação de possíveis melhorias em projetos futuros.

O trabalho de \textcite{alfonso:2006} se refere à criação de um sistema de autenticação em que o usuário recebe uma mensagem desafio em um módulo portátil e a responde, para assim poder utilizar o veículo (funcionamento do motor e partes elétricas). Uma das funções deste dispositivo é o desligamento do veículo caso a central não detecte que o módulo portátil nas proximidades do veículo, se assemelhando às técnicas utilizadas nos alarmes veiculares mais modernos.

Por modularizar os componentes e estes se comunicarem utilizando radiofrequência,
se fez necessária a utilização de criptografia para que as mensagens trocadas não 
fossem interceptadas, interpretadas e posteriormente injetadas, em uma tentativa 
de fraudar o sistema. 

Apesar de descrever textualmente de modo detalhado cada módulo desenvolvido, o trabalho não traz modelagem de requisitos e de projeto baseada em UML, apenas diagramas resultantes da fase de implementação, além disso, o projeto não implementa uma interface amigável.


\section{Contribui\c{c}\~{o}es}

Este trabalho apresenta o desenvolvimento de um sistema de segurança veicular com uso de GPS, cujas principais funcionalidades são: a obtenção da posição atual de um automóvel e a possibilidade de seu desligamento remoto por meio de uma interface Web.

O processo de elaboração do sistema utilizou a metodologia descritas em \textcite{Wolf:2001} voltada a sistemas embarcados em que aplicam-se  conceitos de Engenharia de Software para concepção de plataformas de tempo real. Com isto foram gerados artefatos de modelagem: diagramas UML englobando requisitos, arquitetura e componentes de hardware/software descritos de forma a sintetizar o funcionamento esperado do conjunto e que servem de base para novas implementações.

A escolha e montagem dos componentes de hardware, o software embarcado, protocolos de comunicação escolhidos e frameworks usados podem servir de guia para projetos com funcionalidades que se encaixem na mesma categoria.  

\section{Metodologia}

Foi conduzida uma revisão bibliográfica relativa a projeto de sistemas
embarcados a fim de aplicar padrões de projetos já consagrados e recomendados para a categoria desta implementação, após esta fase de embasamento teórico, elaborou-se a modelagem de requisitos utilizando modelo de casos de uso derivando o diagrama de classes, diagramas de sequências e diagrama de atividades.

A elaboração de um diagrama de componentes foi feita para se obter uma melhor visão do sistema, distiguindo módulo embarcado, módulo Web e interface 
com o usuário. Após isto, foram definidos os componentes de hardware
para construção do módulo embarcado e feito estudo da documentação destes componentes que fora fornecida pelo fabricante. 

Seguido a construção do módulo embarcado, realizou-se a implementação do serviço Web em conjunto com a interface com o usuário (UI), onde foi criada a estrutura que de comunicação entre módulos, realizando fluxos de controle, comando, armazenamento transitório de dados e integração com o serviço de mapas.  

Finalizada a implementação dos módulos, a integração de todas as partes do sistema foi realizada juntamente com testes exploratórios, verificando o comportamento real da aplicação. O módulo embarcado foi instalado em um automóvel real e sua posição mostrada no serviço de mapas foi aferida para validar o funcionamento correto da plataforma, suas limitações de resposta, além da checagem da função de bloqueio. Os testes exploratórios foram registrados em vídeo que se encontra anexado a monografia em mídia digital.

\section{Organiza\c{c}\~{a}o}

O restante texto deste trabalho está organizado da seguinte maneira:

\begin{itemize}
	\item O capítulo 2 apresenta uma descrição teórica sobre os itens utilizados para o desenvolvimento do sistema.
	\item O capítulo 3 descreve os passos do desenvolvimento em si, desde os requisitos, o conceito da solução, a modelagem dos compomentes e finalmente a implementação destes.
	\item O capítulo 4 apresenta os resultados obtidos com a implementação, dados de testes realizados durante o processo de desenvolvimento.
	\item O capítulo 5 descreve as conclusões e trabalhos futuros.
	
\end{itemize}