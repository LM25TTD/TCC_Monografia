\chapter{Considera\c{c}\~{o}es Finais}
\label{cap:consideracoes finais}

\section{Conclus\~{a}o}

A preocupação com a segurança dos próprios bens tem se tornado cada vez mais intensa,
uma vez que atos onde criminosos praticam roubos ou furtos ocorrem com bastante
frequência. No caso de veículos automotores, o proprietário fica sem poder algum para
evitar o crime, já que uma reação pode colocar em risco sua vida e de outros que o acompanham.
Nesta situação se faz necessário o uso de algum mecanismo que permita ao
usuário recuperar sua propriedade minimizando os riscos à sua integridade e de outros que
o estejam acompanhando no momento da ação criminosa.

Neste trabalho foi desenvolvido um sistema de segurança veicular com uso de GPS, composto um módulo de hardware embarcado no veículo e um sistema Web, cujo escopo soluciona o problema citado, permitindo ao usuário rastrear e bloquear seu veículo via Internet. 

Houve a opção pelo uso da plataforma de hardware livre Arduino, pois conta com diversos componentes desenvolvidos para aplicação imediata, documentação disponibilizada na internet e uma comunidade extensa de desenvolvedores. Ao longo da elaboração do sistema foram gerados artefatos documentais que, juntamente com os códigos-fonte, foram disponibilizados publicamente.

O protótipo foi submetido a testes isolados e testes experimentais, este último contemplou a integração do módulo de harware a um veículo, permitindo que uma análise de comportamento em ambiente real fosse realizada. O sistema Web foi submetido a testes de performance e exploratórios, obtendo plena aprovação de seu comportamento.  

O projeto se mostrou economicamente viável mesmo quando utilizados componentes de prototipagem, portanto uma produção em escala industrial derrubaria o custo e permitiria a miniaturização do conjunto do módulo veicular embarcado.

De acordo com o estudo e testes realizados, o sistema final conseguiu cumprir com sucesso a proposta e os objetivos que foram apresentados na fase inicial de projeto, tudo isto dentro do tempo disponível e seguindo o cronograma elaborado na mesma proposta.
 

\section{Dificuldades encontradas}

Ao longo do processo de desenvolvimento alguns entraves foram encontrados, porém alternativas para solução foram aplicadas com sucesso, permitindo a elaboração do sistema completo.

A compra de componentes eletrônicos no mercado local foi um problema devido a ausência de alguns destes para venda, como é o caso de microcontroladores, módulo GPS e módulo GSM. A importação dos produtos foi a solução mais viável, mesmo que o custo financeiro tenha sido elevado.

Conforme citado no capítulo 3, o fabricante do módulo GSM SM5100B não disponibiliza uma biblioteca para comunicação com Arduino, portanto houve a necessidade de se criar tal biblioteca, o que acarretou no investimento de tempo para tal atividade.

A placa Arduino Uno possui uma quantidade de memória RAM bastante limitada, o que gerou a necessidade de diversos ciclos de desenvolvimento e testes (aumentando o tempo de projeto em relação ao estimado inicialmente) até que o firmware fosse otimizado para operar com tal quantidade de memória.

No momento da integração com o veículo, a falta de documentação dos circuitos elétricos do mesmo dificultou a integração do protótipo. Após uma pesquisa em fóruns especializados em alarmes, foi possível localizar o cabeamento a ser alterado para devida integração com o módulo embarcado.

Por fim, a falta de hospedagem gratuita para aplicações Java Web foi um empecilho inicial, superado com criação de um servidor próprio acessível externamente com auxílio da plataforma \textit{no-ip.org}.

\section{Trabalhos futuros}

Existem alguns pontos de melhoria que podem ser aplicados ao sistema atual, otimizando-o e até mesmo gerando um projeto de novo escopo. Como trabalhos futuros sugerem-se algumas propostas:

\begin{itemize}
	\item Desacomplamento da interface com o usuário do sistema Web deixando somente como serviço de \textit{backend}, permitindo a criação de um cliente mobile que consome dados deste seviço.
	\item Registrar os pontos geográficos comuns do veículo, criando um túnel virtual. Caso o veículo se desvie deste túnel virtual um alarme será emitido.
	\item Utilizar uma estratégia de comunicação assíncrona entre o módulo embarcado e o \textit{webservice}, como uso de Websockets na placa Arduino. Esta abordagem pode melhorar o tempo de resposta do sistema.
	\item Criar um circuito de alimentação elétrica de \textit{backup}, mantendo o módulo embarcado funcionando caso haja desligamento da bateria do automóvel.
	\item Monitoramento do acionamento do alarme com envio de SMS ou realização de chamada telefônica ao proprietário em caso de anomalias.
	\item Comunicação direta com celular, possibilitando o bloqueio automático do veículo caso o proprietário leve o aparelho móvel consigo. 
	\item Integração de uma câmera para capturar imagens do interior do veículo e enviá-las a aplicação Web, facilitando a identificação de criminosos.
\end{itemize}

\section{Disciplinas aplicadas}

O desenvolvimento do projeto demandou aplicação do conhecimento de diversas disciplinas integrantes do curso de Engenharia de Computação, as principais foram:

\begin{itemize}
	\item Lógica e linguagem de programação
	\item Circuitos elétricos
	\item Eletrônica analógica e digital
	\item Redes de computadores
	\item Projeto de sistemas embarcados
	\item Sistemas distribuídos
	\item Interface entre usuários e sistemas computacionais
	\item Engenharia de software
	\item Banco de dados
\end{itemize}